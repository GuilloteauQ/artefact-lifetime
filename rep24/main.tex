\documentclass[sigconf,natbib=false]{acmart}
%%
%% \BibTeX command to typeset BibTeX logo in the docs
\AtBeginDocument{%
  \providecommand\BibTeX{{%
    \normalfont B\kern-0.5em{\scshape i\kern-0.25em b}\kern-0.8em\TeX}}}

%% Rights management information.  This information is sent to you
%% when you complete the rights form.  These commands have SAMPLE
%% values in them; it is your responsibility as an author to replace
%% the commands and values with those provided to you when you
%% complete the rights form.
\setcopyright{acmlicensed}
\copyrightyear{2024}
\acmYear{2024}
\acmDOI{XXXXXXX.XXXXXXX}

%% These commands are for a PROCEEDINGS abstract or paper.
\acmConference[ACM REP'24]{2024 ACM Conference on Reproducibility and Replicability}{June 18-20, 2024}{Rennes, France}
%
%  Uncomment \acmBooktitle if th title of the proceedings is different
%  from ``Proceedings of ...''!
%
%\acmBooktitle{Woodstock '18: ACM Symposium on Neural Gaze Detection,
%  June 03--05, 2018, Woodstock, NY} 
\acmISBN{978-1-4503-XXXX-X/18/06}

\usepackage{tcolorbox}
\tcbuselibrary{theorems}
\usepackage{cleveref}
\newtcbtheorem[]{trap}{Pitfall}{colback=black!5,colframe=black!35,fonttitle=\bfseries}{th}
\newtcbtheorem[]{lesson}{Takeaway}{colback=black!5,colframe=black!35,fonttitle=\bfseries}{th}
\newcommand{\repro}{reproducibility}
\newcommand{\Repro}{Reproducibility}
\newcommand{\transpo}{\emph{Transposition}}
\newcommand{\flavour}{\emph{flavour}}
\newcommand{\flavours}{\emph{flavours}}
\newcommand{\ie}{\emph{i.e.,}}
\newcommand{\eg}{\emph{e.g.,}}
\newcommand{\nix}{\emph{Nix}}
\newcommand{\nixos}{\emph{NixOS}}
\newcommand{\nxc}{\emph{NixOS Compose}}
\newcommand{\enos}{\emph{EnOSlib}}
\newcommand{\grid}{\emph{Grid'5000}}
\newcommand{\kam}{\emph{Kameleon}}
\newcommand{\kad}{\emph{Kadeploy}}
\newcommand{\mel}{\emph{Melissa}}
\newcommand{\store}{\emph{Nix Store}}
\newcommand{\ad}{Artifact Description}
\newcommand{\aeval}{Artifact Evaluation}
\newcommand{\adae}{\ad/\aeval}
\newcommand{\todo}[1]{{\color{red}TODO: #1}}
\usepackage{hyperref}
\usepackage{array}
\usepackage{caption}
\usepackage{subcaption}
\usepackage{graphicx}
\usepackage{siunitx}
%\usepackage[table]{xcolor}
\usepackage{multirow}
\usepackage{hhline}
\usepackage{calc}
%\usepackage{tabularx}
\usepackage{fontawesome}
\usepackage[para,online,flushleft]{threeparttable}
\usepackage{booktabs}
\usepackage{longtable}
\usepackage{amsmath,amsfonts}
\usepackage{textcomp}

%\usepackage[backend=biber,style=trad-abbrv,firstinits=true]{biblatex}
\usepackage[
  datamodel=software,
  style=trad-abbrv,
  backend=biber
]{biblatex}
\addbibresource{references.bib}
\usepackage{software-biblatex}

\begin{document}

\title{State-of-practice of \ad s/\aeval s: Are~we~reproducible~yet?}

\author{Quentin Guilloteau}\author{Florina M. Ciorba}
\email{firstname.lastname@unibas.ch}
\affiliation{%
  \institution{University of Basel}
  \country{Switzerland}
}

% \renewcommand{\shortauthors}{Trovato and Tobin, et al.}

%%
%% The abstract is a short summary of the work to be presented in the
%% article.
\begin{abstract}
  \todo{}
\end{abstract}

%%
%% The code below is generated by the tool at http://dl.acm.org/ccs.cfm.
%% Please copy and paste the code instead of the example below.
%%
\begin{CCSXML}
<ccs2012>
 <concept>
  <concept_id>00000000.0000000.0000000</concept_id>
  <concept_desc>Do Not Use This Code, Generate the Correct Terms for Your Paper</concept_desc>
  <concept_significance>500</concept_significance>
 </concept>
 <concept>
  <concept_id>00000000.00000000.00000000</concept_id>
  <concept_desc>Do Not Use This Code, Generate the Correct Terms for Your Paper</concept_desc>
  <concept_significance>300</concept_significance>
 </concept>
 <concept>
  <concept_id>00000000.00000000.00000000</concept_id>
  <concept_desc>Do Not Use This Code, Generate the Correct Terms for Your Paper</concept_desc>
  <concept_significance>100</concept_significance>
 </concept>
 <concept>
  <concept_id>00000000.00000000.00000000</concept_id>
  <concept_desc>Do Not Use This Code, Generate the Correct Terms for Your Paper</concept_desc>
  <concept_significance>100</concept_significance>
 </concept>
</ccs2012>
\end{CCSXML}

\ccsdesc[500]{Do Not Use This Code~Generate the Correct Terms for Your Paper}
\ccsdesc[300]{Do Not Use This Code~Generate the Correct Terms for Your Paper}
\ccsdesc{Do Not Use This Code~Generate the Correct Terms for Your Paper}
\ccsdesc[100]{Do Not Use This Code~Generate the Correct Terms for Your Paper}

%%
%% Keywords. The author(s) should pick words that accurately describe
%% the work being presented. Separate the keywords with commas.
\keywords{Do, Not, Us, This, Code, Put, the, Correct, Terms, for,
  Your, Paper}

\received{12 February 2024}
\received[revised]{12 March 2009}
\received[accepted]{5 June 2009}

%%
%% This command processes the author and affiliation and title
%% information and builds the first part of the formatted document.
\maketitle

%% PAPER STARTS HERE -----------------------------------------------------------------------------------------------

\section{Introduction}

\todo{this is some text from my thesis, and it should be cleaned}

The scientific community as a whole has been traversing a \repro\ crisis for the last decade.
Computer science does not make an exception\ \cite{randallIrreproducibilityCrisisModern2018,baker500ScientistsLift2016}.

The \repro\ of the research work is essential to build robust knowledge, and it increases the reliability of results while limiting the number of methodology and analysis bias.
In 2015, Collberg et al.\ \cite{collberg_repeatability_2015} studied the \repro\ of 402 experimental papers published in *system* conferences and journals.
Each studied paper linked the source code used to perform their experiments. 
On those 402 papers, 46\% were not reproducible.
The main causes were:
\emph{(i)} the source code was actually not available,
\emph{(ii)} the code did not compile or did not run,
\emph{(iii)} the experiments required specific hardware

To highlight the reproducible research works, several publishers (like ACM or Springer) set up an artifact evaluation of a submission.
This peer review process of the experimental artifact can yield one or several badgers to the authors based on the level of \repro\ of their artifacts.

The term \repro\ is often used in a broad sense and gathers several concepts.
The definitions that we will use in the rest of this thesis are the ones defined by ACM for the validation of the submitted artifacts\ \cite{acm-badges}.
It is composed of three levels of \repro:

\begin{enumerate}
\item \emph{Repeatable}: the measures can be obtained again by the people at the origin of the work.
\item \emph{Reproducible}: the measures can be obtained again by people who do not belong to the original work and with the original artifact of the authors.
\item \emph{Replicable}: the measures can be obtained again by people who do not belong to the original work without the original artifact.
\end{enumerate}

The evaluation of artifact is a crucial point which allows guaranteeing the reproducibility of the experiments and the results.
However, this reproducibility is not sufficient.
Even if being able to reproduce an experiment is proof a scientific validation, the experiment and its environment are often too rigid to be extended by a third party, or even by the authors themselves. 
We believe that the notion that should be pushed by the community is the \emph{reproducibility with variation}.
By "variation" we mean that a third party is able to easily modify the environment of the experience to continue the research.
This means that the hardware and software environments as well as the experimental scripts must be correctly defined and can be modified easily.

This section focuses on the software environment.
For a global vision of the reproducibility problems, the readers might be interested in\ \cite{ivie2018reproducibility}.

An experiment without the environment in which it was executed makes it much more difficult to reproduce.
Indeed, side effects from the environment can happen and change the results of the experiment.
It is easy to forget to include in the software environment an element which impacts the performance of the experiment.
The performances, but also the results of a simple C application can depend on the compilation options\ \cite{stodden2018assessing} or also from the quantity of UNIX environment variables\ \cite{mytkowicz2009producing}.

Most of the current solutions in terms of "\repro" fall under the storage of artifacts (system images, containers, virtual machines) and replay of experiments\ \cite{rosendo2020e2clab, brammer2011paper, brinckman2019computing}.
Even if this is an important part of the \repro\ spectrum, nothing guarantees that the software environment can be re-built in the future, and thus nothing guarantees that the experiments can be re-run if the artifacts disappear. 


The step of artifact evaluation for the conferences is done soon after their initial construction.
It is thus very probable that the construction of the artifacts will be executed in a similar state of the packages mirrors (\texttt{apt}, \texttt{rpm}, etc.).
However, what will happen when someone will try to rebuild the environment in 1 year? 5 years? 10 years?
The objective of science is to base itself on robust works to continue to go forward (\emph{Stand on the shoulders of giants}).
This vision of "\textbf{short term reproducibility}" is a major obstacle to scientific progress and is in complete opposition to the science philosophy.


We think that the notion that should be highlighted is the concept of \textbf{variation}\ \cite{mercier2018considering, feitelson_repeatability_2015}.
This means allowing a third party to use the environment defined for an experiment in order to investigate another research idea.
An example of variation would be to change the MPI implementation used in an experiment (\eg\ MPICH instead of OpenMPI).
Being able to introduce such a variation is only possible if the initial environment is correctly defined.

%
% /////////////////////////////////////////////////////////////////////////////////////////////////////
\section{Context}

\todo{}

We will modify a bit the definition of "\repro" as in \cite{rougier2019rescience}:
an experiment is \emph{reproducible} if the source code, the raw data, the analysis scripts are available, their usage described sufficiently for someone to re-perform the experiments and analyis, and the access to the used experimental platform is open.
\todo{to discuss more}


\subsection{Tools}

Prova \cite{guerrera2019reproducible}
BenchOpt \cite{moreau2022benchopt}
UMLAUT \cite{umlaut}
Benchmarking Crimes \cite{van2018benchmarking}
Kheops \cite{rosendo2023kheops}

Cascad \cite{perignon2019certify}

\subsection{Journals}


\todo{Journal of open research software \cite{hong2013software}}

The last decade has seen the creation of independent online scientific journals to reward software and reproducibility.

% JOSS \cite{smith2018journal}
The most popular example is probably the \emph{Journal of Open Source Software} (JOSS)\ \cite{smith2018journal}.
This journal publishes article about open source research software.
The reviewing process includes a thorough inspection of the source code, the documentation of the software, as well as a runthrough of some examples.
Reviews are open and hosted online as Github issues.

%IPOL \cite{colom2015ipol}
In the field of Image Processing, the online journal \emph{Image Processing On Line} (IPOL)\ \cite{colom2015ipol} requires the authors to implement the algorithms proposed in their paper and to make the impementation available through an online demonstration fo reader to explore and play with.
This requirement forces the authors to share their code alongside their paper.
IPOL noted that it also helped improve their algorithms, as actually implementing the algorithms might raise some undetected edge-cases.

% Papers with code \cite{paperswithcode}
\emph{Papers with code}\ \cite{paperswithcode} is a website that references Machine Learning academic papers that shared their code/datasets.
Since 2015, they checked the code availability for the open access Machine Learning papers.
They exihib that the code availability of the Machine Learning papers has gone from less than 10\% in 2015, to a relatively constant 30\% since 2019.
\todo{berk?}

% Rescience \cite{rougier2019rescience}
The \emph{ReScience} online journal\ \cite{rougier2017sustainable, rougier2019rescience} aims to encourage the replication (successful or not) of already published research work.
Similarly to JOSS, the submissions and reviews are open and managed through issues and pull requests on Github.
\todo{more}

\subsection{Surveys}

Propogation of ML research  \cite{kang2023papers}
Survey 4R \cite{hernandez2023repeatability}
Quality indicator \cite{castell2024towards}

%Survey Sascha \cite{hunold2015survey}
In 2015, Hunold\ \cite{hunold2015survey} conducted a survey among the participants of the EuroPar conference to assess the vision of the parallel computing community on the questions of reproducibility. 
When asked about the main reasons for not making the source code/raw data/data processsing available, participants answered that: "\emph{it is irrelevant because evolution is too fast}" (90\%), "\emph{it is not rewarding}" (87\%), "\emph{I want to retain a competitive advantage}" (84\%).
The second most popular answer is quite interesting as \aeval\ processes were not very popular at the time of the survey (the very first one was in 2011 at ? \todo{check, it's in my thesis}).
The work done by publishers with the badging system aimed to reward authors for sharing their artifacts.
It would be interesting to reconduct the survey today to see the impact on the \aeval\ on this question, as we believe that reproducibility and its challenges have now much more visibility than ever.

\subsection{\adae\ and badges}

\cite{winter2022retrospective}
\cite{hermann2022has}
\cite{hermann2020community}

\cite{beller2020will}



%
% /////////////////////////////////////////////////////////////////////////////////////////////////////
\section{Recommendations for \ad s}

\todo{look at the recommendation of the conferences for the \adae, and blog posts, and summarize}

%
% /////////////////////////////////////////////////////////////////////////////////////////////////////
\section{State-of-practice and limits of \ad}\label{sec:sop}

\input{tables/summary_conferences}

In this Section, we surveyed \todo{N} papers from the top conferences in distributed systems and system in 2023.
All of the considered conferences had an \adae\ process for the \emph{accepted} papers.
This \adae\ process usually consists of a first phase where the authors write an \ad\ as an appendix of the paper to show how to get and use the artifact, how to install the dependencies, what are the different experiments and their estimated time, etc.
This \ad\ section is usually a page long (in a double column layout).
The most popular practice is for authors to give a link to a more detailed description of the artifact.

Table \ref{tab:table:paper_confs} summarizes the number of papers evaluated for each conferences, as well as the number of artifact sections.


\begin{figure}
  \centering
  \includegraphics[width=0.5\textwidth]{figs/does_paper_with_badge_has_artifact_section}
  \caption{Number of papers which earned at least one badge which contain an \ad\ in the pdf. We observe that a significant amount of papers with at least a badge do not share their \ad\ in the proceedings version. \todo{remove plot and just put the percentage in the text}}\label{fig:}
\end{figure}

A surprising realisation is that not all the papers with the "Open" badge have the artifact section in the proceeding version.
We suspect that either the author forgot, or refused, to include the section in the final version of the paper.
We believe that \textbf{the \ad\ is for the artifact reviewer, but also a future reader of the paper.}
Excluding this section of the final version should not grant the "Open" badge.

\subsection{Methodology}

We selected conferences of 2023 with an \ad\ process, and looked at their proceedings.
From the proceedings, we surveyed all the papers and, for each paper, noted:

\begin{itemize}
  \item How many reproducibility badges, and which, did the paper received
  \item If the paper has an \ad\ section
  \item If the paper shares the URL of the source code (it does not have to be in the \ad), and if the URL is still valid
  \item How the experiments were performed (\eg\ local machines, shared testbeds, proprietary machines, supercomputer, simulation, etc.)
  \item How was the source code shared (\texttt{git}, Zenodo, Software-Heritage, or combination of solutions)
  \item If the source code has been shared with \texttt{git}, we record the number of commits, and check if a precise commit has been specified by the authors
  \item How the software environment was shared
\end{itemize}

We spent around 5 minutes, and no more than 10 minutes, per paper to collect the aforementioned information.


In the following of this Section, we study the \ad s on four characteristics: how is the source code shared (Section \ref{sec:sop:src}), where are the experiments executed (Section \ref{sec:sop:expe}), how is the software environment described and shared (Section \ref{sec:sop:sw}), and finally the worflow of experiments (Section \ref{sec:sop:workflow}).

\paragraph{Disclaimer}

The data presented in this paper has been collected by hand through a long and tedious process.
Eventhough we did our best to evalute correctly all the surveyed papers, we do not believe that there are no mistake.
But we are certain that if there are errors, they will not affect significantly the conclusions of this paper.

\subsection{Source code}\label{sec:sop:src}

\begin{figure}
  \centering
  \includegraphics[width=0.5\textwidth]{figs/how_repo_shared}
  \caption{Methods used by the users to share their artifact source code. The state-of-practice is dominated by \texttt{git} URLs, Zenodo archives, and a combination of the two.}\label{fig:how_repo_shared}
\end{figure}

\todo{update order based on new papers}

The \ad\ gives a link to where to download the artifact.
Figure \ref{fig:how_repo_shared} shows how the code of the artifact was shared in the \ad.
The majority of papers simply gives the URL of a \texttt{git} repository.
Some paper shared their code with Zenodo\ \cite{zenodo} or Figshare\ \cite{figshare}, and some shared with both a \texttt{git} URL and Zenodo.
A minority used Software-Heritage\ \cite{swheritage}.

Some authors used \texttt{anonymous.4open.science}\ \cite{anonymous_github} which allows users to share with the reviewers an anonymous copy of a public Github repository.
This is particularly useful for double blind reviews.
However, for all the papers surveyed, all the links to this service were dead, and there was no way to find the original \texttt{git} reprository.
We believe that the links simply expired, which is problematic for the usage of future researchers.
Moreover, as long as the \aeval\ process does not count in the accept/reject decision, having a double blind review for the \adae\ only limits the ability of the reviewers and authors to communicate.
If the results of the \aeval\ will be taken into account for the decision, then the community should investigate ways to perform the \aeval\ in a double blind fashion.
An easy solution would be to use tools such as \texttt{anonymous.4open.science} for the review, and then substitute in the camera ready version of the paper the URL for a more persistant one.

When authors shared only with a \texttt{git} URL, they almost never mention the commit used for their experiments (see Figure \ref{fig:was_commit_fixed}).
Such a solution might be satisfactory for the \aeval\ as the delay between the submition of the paper and the evalation of its artifact is short enough to keep the code in the same, or a similar state.
However, for future researchers aiming to build upon these artifacts, it is nearly impossible to know which version of the code was used.
Moreover source code hosted on forges might not be available forever.
Authors could decide to delete or rename their repository, invalidating the URL given in the \ad.
Worse, the entire source forge might need to close. \todo{ref about gforge inria?}
\todo{a word on publiware?}

\todo{link to Section 3}
One solution proposed by the reproducibility guidelines of the conferences is to archive the code via Zenodo or Figshare, and then reference the DOI generated by these archive websites in the \ad.
This has the advantage of freezing the source code in a permanent fashion, and allow future usage of the code.
However, storing source code on Zenodo has a simple drawback: there is no code exploration in the browser.
From the point of view of future researchers, having to download locally potentially large archives from Zenodo to be able to explore a few files of the source code is cumbersome and introduces friction.
A better solution woulbe be to explore the artifact via a simple web UI.
Archiving can also break the link between the original \texttt{git} URL if not archived correctly.
Zenodo has an integration with Github\ \cite{github_zenodo} which allows to archive releases of a reprository.
This is why some authors share both the \texttt{git} URL and a Zenodo archive.
If the link between the repository and the Zenodo archive breaks (\eg\ \texttt{git} repository disappears), then future researchers are left with a single commit of the source code, and all the history of the project, which contributes to the understanding and extensibility of the project, is lost.
Some artifact shared via Zenodo are actually an archive of a \texttt{git} repository and include the \texttt{.git} folder.
\todo{more.}
Zenodo and Figshare are more adapted to archive datasets and binaries, not source code.
This heavy usage of Zenodo and Figshare comes from the requirements from the artifact reviewing committee to have identified and citable software, and this goes through giving a DOI to the artifact.
However, these solutions are more appropriate for raw data or binaries, but not software \cite{alliez2019attributing}.

A more appropriate solution is to use Software-Heritage\ \cite{swheritage, di2017software}.
Similarly to Zenodo, it offers a permanent storage of source code, with the same interface as usual source forges (\eg\ Github, Gitlab, etc.).
It also refers the orignal source, so that future researcher can see it if still available.
Authors give an unique identifier for the revision. 
\todo{share the swhid of this repo}

\todo{what we want: history + check new version + web UI}

\begin{figure}
  \centering
  \includegraphics[width=0.5\textwidth]{figs/was_commit_fixed}
  \caption{Number of papers that did precise an exact commit to use when the authors shared their source code \emph{only} via a \texttt{git} URL. We observe that the crushing majority do not precise the commit, and thus break the traceability of the artifact. \todo{remove plot and put percentage in the text}}\label{fig:was_commit_fixed}
\end{figure}

\begin{figure}
  \centering
  \includegraphics[width=0.5\textwidth]{./figs/number_commits_repo.pdf}
  \caption{Cumlative distribution function of the number of commits in the repositories shared in the \ad s}\label{fig:number_commits_repo}
\end{figure}

One thing that we observed during this survey is the very low number of commits in the repositories linked in the \ad s when shared with \texttt{git} (\ie\ \texttt{git}, \texttt{git+zenodo}, \texttt{git+figshare} on Figure \ref{fig:how_repo_shared}).
Figure \ref{fig:number_commits_repo} shows the cumulative distribution function of the number of commits in the repositories shared in the \ad s when the source code was shared via \texttt{git}.
\todo{update this final data: We can see that more than a third have less than 10 commits.}
These repositories appear to be a "dump" of the source code with some extra commits for documentation.
\todo{dump from where?}
Such practices do not allow reviewers and future researchers to explore the "true" history of the project.
The same comments as for a standalone Zenodo archive apply here.
It also makes it suspicious about the good practices of the authors in terms of experimentations (traceability).
This problem might come from the fact that the \aeval\ process, and reproducibility, is only a second though for authors.

\begin{lesson}{Sharing source code}{}
  A shared \texttt{git} URL might not be available in the future.
  Archiving via Zenodo is better but introduce friction for future exploration.
  Using Software-Heritage appears to be the best available solution to permanently share source code.
\end{lesson}

\todo{cloud/google drive}

\subsection{Experimental setup}\label{sec:sop:expe}

One important point about reproducing experiments is the hardware used.
Any experiment that exhibts a particular behavior or performance evaluation should have a description of the hardware. 

\begin{figure}
  \centering
  \includegraphics[width=0.5\textwidth]{./figs/experimental_setup.pdf}
  \caption{Experimental setup used in the survey papers. Most of the authors use local machines at their disposal. Some authors also use supercomputers to experiment on state-of-the-art systems. More concerning is the number of papers relying on proprietary platforms such as Amazon Web Services, Google Cloud, Microsoft Azure. Finally, a small proportion of the papers uses shared testbeds such as Chameleon, CloudLab, or Grid'5000.}\label{fig:experimental_setup}
\end{figure}

Figure \ref{fig:experimental_setup} depicts on which platforms the experiments of the considered papers were executed.
Most of the experimental platforms were local machines ("homemade"), but the description of the machine (\ie\ CPU, GPU, disk, etc.) were given.
This still makes it difficult for the reviewers to find the exact hardware, or the closest they can.
In some \ad s, we observed that authors were gave access to their local machines by giving the IP address and the password to connect.


A better solution would be to use open and shares platforms, also called \emph{testbeds}.
Chameleon\ \cite{chameleon}, Grid'5000\ \cite{grid5000}, or CloudLab\ \cite{cloudlab} are example of such testbeds.
\todo{more \cite{nussbaum2017testbeds}}

\todo{SC24 will use Chameleon for \aeval}

In practice testbeds are not used a lot.
However, proprietary solutions are.
Authors rely on platforms such as Amazon Web Services, Microsoft Azure , Google Cloud, etc.
Even if this allows the reviewers to get access to probably similar machines (in the short term), it hides the experiments, and thus their reproducibility, behind a paywall, which goes against the principle of Open Science.
Using proprietary platforms also raises the question of who should pay to reproduce the results of the authors?
Some authors using such platforms did write in their \ad\ the estimated monetary cost of rerunning the experiments.

Similarly, some papers (in particular for the SuperComputing conference (\texttt{sc23})) were using supercomputer to run their experiments.
While supercomputers are at the bleeding edge of technology, having access to such system is restrictive and can take several weeks or month before getting accepted.
\todo{berk}

\begin{lesson}{Experimenal setup}{}
  The majority of \ad s use machines that are difficult to get access to (local, supercomputer, or proprietary). 
  The testbeds are under representated in the state-of-practice, but appear to be the the way forward \cite{nussbaum2017testbeds} (\todo{berk}).
\end{lesson}

\subsection{Software environment}\label{sec:sop:sw}

\begin{figure*}
  \centering
  \begin{subfigure}{0.49\textwidth}
    \centering
    \includegraphics[width=\textwidth]{figs/how_packaged.pdf}
    \caption{Tools and technologies the authors used to generate/package their software environment. The majority of the artifacts did not use any tool. The rest used virtualization tools (\eg\ containers or virtual machines), the most used being Docker.}\label{fig:techno}
  \end{subfigure}
  \hfill
  \begin{subfigure}{0.49\textwidth}
      \centering
      \includegraphics[width=\textwidth]{figs/image_cache_bin.pdf}
    \caption{When authors used virtualization tools, most of them provided the image in a binary cache to avoid the reviewers to build the image from source. However, only a small fraction of these image where archived in a long-term binary cache (always Zenodo). The most concerning observation is that we could not find the recipe of the image in about half of the artifacts.}\label{fig:cache_bin}
  \end{subfigure}
  \caption{Tools and technologies used to generate and package the software environment for the \aeval\ (Figure \ref{fig:techno}), and state of the image and its recipe in the case of the use of virtual tools (Figure \ref{fig:cache_bin}).}\label{fig:techo_cache}
\end{figure*}

\begin{figure*}
  \centering
  \includegraphics[width=\textwidth]{figs/sw_envs}
  \caption{Techniques used to share the software environment in the \ad s. Note than an paper can use several of these techniques.}\label{fig:sw_envs}
\end{figure*}

After downloading the correct version of the code on the correct platform, reviewers must set up the correct software environment to execute the experiments.
Figure \ref{fig:sw_envs} shows the different techniques used to describe and share the software environment in the \ad s.
Note that the authors can use \emph{several} of these techniques in their \ad.

In the following of this Section, we go through the methods observed to share the artifact software environment, and discuss their reproducibility.

\subsubsection{Images}

Figure \ref{fig:techno} shows which tools were used to capture the software environment of the experiments.
Most of the \ad\ do not use any particular tool.
But some use virtualization tools such as container and virtual machines.

\todo{also chameleon images}

Usually, the capture of the entire software stack goes through it encapsulation in a system image.
This image can then be deployed on machines to execute the experiments.
A way to generate a system image is to start from a base image, deploy this image, execute the commands required to set up the desired environment, and finally compress the image.
Platforms such as \grid\ \cite{grid5000} and Chamelon\ \cite{chameleon} propose to their users such tools (\texttt{tgz-g5k}\ \cite{tgz-g5k} and \texttt{cc-snapshot}\ \cite{cc-snapshot} respectively).
In the context of repeatability and replicability, if the image stays available, then this way to produce system images is adequate at best.
But, concerning the traceability of the build, one cannot verify the commands that have been used to generate the image, and thus relies completely on the documentation from the experimenter.
Moreover, such images are not adapted to be versioned with tools like \texttt{git} as they are in a binary format.
In the situation where the image is no longer available, re-building the exact image is complex and the precise introduction of variation is utopian.

Figure \ref{fig:cache_bin} depicts the state of the images for the \aeval.
We observed that the majority of the authors who use an image make it available in a binary cache such as DockerHub. 
However, DockerHub does not offer permanent storage of the images, or the authors could push another image on top of the previous one, thus losing the traceability for the experiments.
Another solution is to archive the image in a long-term binary cache such as Zenodo.
However, it is not often done by the authors.

A better approach to generate and share images is via \emph{recipes}.
Those recipes, like \texttt{Dockerfile}s for Docker containers or Kameleon\ \cite{ruiz_reconstructable_2015} recipes for system images, are a sequence of commands to execute on a base image to generate the desired environment.
The text format of recipes make then much more suitable to version, share, and reconstruct them.
These base images have often several versions, which are identified by labels called \emph{tags}.
In the case of Docker, the tag of the latest version is often called \texttt{latest}.
Basing an environment on this tag breaks the traceability, and thus the reconstruction of the image itself.
Indeed, if a newer version is available at the time of a future rebuild of the environment, then the image will be based on this newer version and not the original version.
Another important question is to know if the base image and all the version are themselves reconstructive, and if it is not the case, what is the permanence of the platforms hosting those images?
For instance, the lifetime of \texttt{nvidia/cuda} Docker image is 6 months, after 6 months, the administrators delete the images\ \cite{nvidia_cuda_lifetime}.
However, in Figure \ref{fig:cache_bin}, we see that more than \todo{half} of the \ad\ using an image do not share the recipe, or we could not find the recipe, to inspect or rebuild the image.
This means that if the image is not in a binary cache, then it is impossible to rebuild it.


\subsubsection{List of package versions}\label{sec:sop:sw:list}

One of the popular approach to share the software environment is to simply list the dependencies of the artifact.
We observed several levels to such this listing approach.
The first level is to only give the name of the dependencies.
Then, authors can give a minimum version to use (\eg\ \texttt{gcc >= 10.0.0} \todo{check if makes sense}).
Finally, the most popular approach is to give for all the dependencies the version used.

Listing all the dependencies by hand raises several important questions.
Are actually \emph{all} the dependencies listed?
How to get another system in the same state?
What about the dependencies of the dependencies, etc?

\todo{more?}

\begin{lesson}{Listing dependencies}{}
  Simply listing the used packages is not enough to regenerate the correct software environment.
\end{lesson}

\subsubsection{Package managers installation commands}

Another popular way to describe the software environment is to list installation commands to the package manager (\eg\ \texttt{apt}, \texttt{yum}).
\todo{a word on what is a package manager}
Most of the time these commands look like:

\begin{verbatim}
sudo apt-get update
sudo apt-get install packageA packageB
\end{verbatim}

The question that arises is: which is the version of the installed packages?
Indeed, by calling \texttt{apt-get update} (or equivalent for the others package managers), now the environment depends on the state of the mirror of the package manager at the time the author did the experiments.
In the case of the \aeval, the mirror might not change "too much" between the time of the experiments and the review.
However, there is a very low probability that 5 or 10 years in the future, the mirror will be in the same state, and the installed versions will be the exact same as for the experiments of the authors.
This approach also implicitely defines a dependency on the operating system that needs to be used.

There are "workarounds" to be sure that the installed packages via classical package managers will be the expected ones. 
One of them is to use a \emph{snapshot} of the mirror\footnote{Example for \texttt{debian}: \url{http://snapshot.debian.org/}}.
These snapshots are a dump of the mirror at a given time and users can then install packages from these snapshots throught the usual interface of the package manager.
However, even using snapshots can cause reproducibility issues.
Indeed, what if the package installed from the snapshot creates a conflict with a package already installed on the system?
This is especially an issue for Filesystem Hierachy Standard (FHS) based systems (\eg\ Debian-based distributions) where all the binaries and libraries are stored into \texttt{/usr/bin} and \texttt{/usr/lib}.
For example, what happen to the already installed packages if the artifact requires to install an old version of the \texttt{glibc}? 

\todo{a word on variation}


\begin{lesson}{Classical package managers}{}
  Installing dependencies via classical package managers (\eg\ \texttt{apt}, \texttt{yum}) creates a dependency on a uncontrollable state: the state of the mirror of the package manager.
  Freezing the state of the mirror introduces new problems of compatibility with the underlying system and disables any introduction of variation.
\end{lesson}

\subsubsection{\texttt{pip} and \texttt{conda}}

In the case where the software environment contains only Python packages, freezing the dependencies with \texttt{pip} (\texttt{pip freeze}) is not enough.
\texttt{pip} only describes the Python environment, and ignores the system dependencies that numerous packages have. 
For example, freezing an environment containing the \texttt{zmq} Python package will not freeze the ZeroMQ system package installed on the system.  
Even if re-creating a Python environment from a \texttt{requirements.txt} is simple, installing a list of system packages with specific version is on the other hand much more complex.

In the best case, the repository include \texttt{requirements.txt} that list all the Python dependencies with the \emph{exact} versions.
However, in practice, we observed the same issues as presented in Section \ref{sec:sop:sw:list} where authors provide a list of dependencies without version, or with a loose version.

\todo{more on pip, pythonpath, conda}

\subsubsection{Downloading from the outside world}

A common practice when needed to install a package that is not available through the classical package managers, is to install it from source.
For this, authors indicate in the \ad\ how to download the dependency, and how to build it.
However, when cloning a \texttt{git} repository or downloading an archive via \texttt{wget}/\texttt{curl}, a common error is to not specify the commit to use.
If no commit is specified, \texttt{git} will use the latest commit of the main branch, which could be completely different at the moment of the artifact review and in 10 years.

Moreover, the downloaded \texttt{git} repository could disappear in the future, and thus cloning from Software-Heritage would be more robust than cloning from a forge (\eg\ Github, Gitlab).

Another important point is to check that the downloaded object is indeed the expected one.
This can be done by checking the cryptographic hash of the downloaded object and compare it to the expected one.
Among the surveyed papers, we observed this practice only once.

\begin{lesson}{Content of downloaded objects}{}
Every object coming from the outside of the environment must be examined to be sure that it contains the expected content. 
It is more important that the building of the environment fails if the content differs from the expected one, rather than the environment silently building with a different content.
\end{lesson}

\subsubsection{Modules}

A popular way to manage a software environment on HPC systems is through \emph{Modules} \cite{modules}.
Modules allow users to change their environment by "loading" and "unloading" packages, and allow to simply manage different versions of applications. 
Under the hood, modules change the \texttt{\$PATH} environment variables.
Modules are however mainly maintained by the administrators of the system, and are system-specific (\eg\ compile MPI with special optimizations for the underlying system).
Thus sharing a module-based environment between two systems might be impossible.
Moreover, as the \texttt{modulefiles} are managed by the administrators, they do not have an infinite lifetime, and might be unavailable in the future.
Modules are also helpful for administrators to limit the applications that can be run by users.
Modules can be generated with Easybuild \todo{ref + more}

\subsubsection{Spack}

Tools such as Spack\ \cite{gamblin_spack_2015} have a similar approach as \texttt{pip} but also for all the system packages and their dependencies.
It is possible to export the environment as a text file and to rebuild it on another machine.
However, the produced environment might not be completely identical.
Indeed, Spack uses applications that are already present on the machine to build the packages from the sources.
Especially, Spack assumes the presence of a C compiler on the system, and will use this C compiler to build the dependencies of the environment.
Hence, if two different machines have two different C compiler then the resulting environment could differ from the desired environment.
One clear advantage of Spack is the ease to introduce a variation in an environment through the command line.
Spack can also be run as a non privileged user and does not require the approval of the system administrators.
Spack will download and build dependencies into a folder located in the user's \texttt{HOME}.
However, a drawback on HPC system is that this directory consumes a lot of storage quota and inodes.

\subsubsection{Vendoring}

One way to make sure to use the correct dependencies is to "vendor" them.
This means having a copy of the dependencies source code which is build from source.
However, this approach has its limits has it does not capture all the dependencies (\eg\ C compiler).

\subsubsection{Nix (Functional package managers)}

Tools such as Nix\ \cite{dolstra_nix_2004} or Guix\ \cite{courtes_functional_2013} fix most of the problems described in the previous sections.
Nix and Guix share the similar concepts, in the following we will focus on Nix.

Nix is a pure functional package manager for the reproducibility of the packages.
A Nix package is defined as a function where the dependencies of the packages are the inputs of the function, the body of the function contains the instructions to build the package.
The building of the packages is done in a \emph{sandbox} which guarantees the build in a strict and controlled environment.
First, the sources are fetched, and the content verified by Nix.
If the hash of the sources differs from the expected hash, Nix stops the building of the package and yields an error.
Nix fetches also the dependencies and recursively.
The build commands are then executed in the sandbox with the environment defined by the user.
At this stage, no network access or access to the file system is possible.


Nix can generate environments that can be assimilated as multi-languages \texttt{virtualenv}s.
But it can also create containers images (Docker, Singularity, LXC, etc.), virtual machines, or full system images.
The process of building an image with classical tools (\texttt{Dockerfile}, Kameleon recipe, etc.) is often iterative and arduous.
Defining an image with Nix is done in a \emph{declarative} fashion.
This has the advantage of making the building of the image faster when modifying an already built recipe\ \cite{nxc}.
It also avoids the annoying optimization of the order of operations, frequent when building from a \texttt{Dockerfile}\ \cite{docker_cache}. 
As Nix packages are functions, introducing a variation means changing an argument when the function is called.

Systems like Debian store all the packages in the \texttt{/usr/bin} and \texttt{/usr/lib} directories.
This ordering can lead to conflicts between different versions of the same library, and it thus limits the introduction of variation in the environment without breaking the system.
On the other hand, Nix creates one directory per package.
Each directory name is prefixed by the hash of its sources.
Hence, if a user wants to install a different version of an already installed package, the sources will be different, thus the hash will be different, and Nix will then create a new directory to store the new package.
Those individual directories are stored in the \emph{Nix Store} located at \texttt{/nix/store}, in a read-only file-system.
The advantage of this fine-grained isolation method, is the \emph{precise} definition of the `\$PATH` environment variable to manage software environments.

The definition of packages through function also eases their sharing and distribution.
There is a large base of package definition done by the community, called \texttt{nixpkgs}\ \cite{nixpkgs}.
Users can easily base their new packages, or environment on those definitions.
It is also possible for independent teams and research groups to have their own base of packages.
Guix-HPC\ \cite{guix-hpc}, NUR-Kapack\ \cite{kapack}, or Ciment-channel\ \cite{ciment_channel} are examples of independent packages base for HPC and distributed systems.

A \textbf{\nix\ system profile} defines the configuration of the system (packages, \texttt{initrd}, etc.).
Among many features, a profile can define filesystems such as NFS and mount them automatically at boot time.
\nixos\ extend the ideas of \nix\ to the entire operating system.
A \nixos\ image can contain several profiles and \nix\ can switch between them by modifying symbolic links and restarting services via \texttt{systemd}.

\paragraph{Limits of Functional Package Managers}

Even though tools like Nix and Guix greatly improve the state of \repro\ for software environments, it is still possible to go wrong and make a package impure or to depend on an exterior state.
Nix is currently addressing this issue with the experimental feature \emph{Flake}\ \cite{flakes}.

To ensure the \repro\ and traceability of an environment, Nix requires that all the packages and their dependencies have their source code open and that the packages are packaged with Nix.
This could seem limiting in the case of proprietary software where the source code is unavailable (Intel compilers for example).
It is still possible to use such proprietary packages with the \texttt{impure} mode of Nix, but it breaks the traceability and thus the \repro\ of the software environment. 


The construction of the packages in a sandbox goes through an isolation mechanism of the file-system using \texttt{chroot}.
Historically, this feature is only available to users with \texttt{root} privileges.
But in the case of computing clusters, this kind of permissions greatly limits the adoption of Nix or Guix.
However, the \emph{unprivileged user namespace} feature of the Linux Kernel allows users to bypass this need of specific rights in most of the cases.

As Nix needs to recompile from source the packages are not available in its binary cache, it is possible that a future rebuild is impossible if the host of the source code disappear\ \cite{blinry}.
However, as Software Heritage now performs frequent archives of the open source repositories, it should be possible to find the sources of interest if needed.

These tools also require a change of point-of-view in the way of managing a software environment, which might make the learning curve intimidating.


\subsection{Workflow managers}\label{sec:sop:workflow}

\todo{more, make, et al, scalable, robust, can execute subsect of the workflow, can interact with the batch scheduler\ \cite{snakemake-executor-plugin-slurm}}

Even if we did not record this information during the survey, a stricking realization is that almost no \ad\ made use of a workflow manager to run the experiments (maybe 1 out of all the  surveyed papers).
There are two main ways that the authors describe the workflow: very long \texttt{bash} scripts or a \texttt{README} file that require to copy-paste the commands.
Some commands are sometime directly included in the LaTeX, which makes it hard to read and to copy-paste.

As experiments in distributed computing can be quite expensive to run (especially if one needs access to a supercomputer or proprietary cloud), having the possibility to run a subset of the workflow is crucial.
For instance, a reviewer or future researcher might want to rerun only the analysis of the data from the artifact (dataset that has been stored on Zenodo), or maybe to add a new combination of parameters. 

Workflow managers \cite{wratten2021reproducible} such as Snakemake \cite{koster2012snakemake}, NextFlow \cite{di2017nextflow}, or Common Workflow Language\ \cite{amstutz2016common} based solutions (\eg\ Guix Workflow\ \cite{strozzi2019scalable}, or Toil\ \cite{vivian2017toil}) has become a standard in bioinformatics to run complex pipelines.

\begin{lesson}{Workflow managers}{}
  The workflows described in the artifacts either rely on manually copy-pasting commands from \texttt{README} files, or executing fragile \texttt{bash} scripts.
  The community would \emph{greatly} benefit by adopting workflow managers\ \cite{wratten2021reproducible}.
\end{lesson}

%\subsection{Conclusion}\label{sec:sop:conclu}
%
%From what we observed in this Section, we can conclude that the state-of-practice of \adae\ is not yet up to the state where it could be used for future researchers.
%
%Proposition: the definition of a new badge assessing of the lifetime of \adae


%
% /////////////////////////////////////////////////////////////////////////////////////////////////////
\section{A new badge for artifacts}

\begin{table*}
  \caption{\label{tab:lifetime}Grading framework for evaluating the \emph{lifetime} of an artifact.}
  \centering
  \begin{tabular}[t]{l p{18em} p{9em} p{22em}}
  \toprule
    Grade & Source Code &  Experimental Setup & Software environment \\
  \midrule
    1/4 & Only \texttt{git} URL with a \emph{fixed} commit, or only Zenodo archive & Proprietary platforms & Vendoring or \emph{precise} download of dependencies \\
    2/4 & \texttt{git} URL and Zenodo archive of a \emph{release} & Local machines & Docker/VM with recipe and long-term storage of the image \\
    3/4 & \texttt{git} URL and Zenodo archive of the repository \emph{with} the history & Supercomputers & Spack \\
    4/4 & Software-Heritage & Testbeds/Simulation  & Nix(OS) / Guix \\
  \bottomrule
  \end{tabular}
\end{table*}

We believe that the current badging system is missing one important aspect of the quality of the artifacts: their \emph{lifetime}.
By \emph{lifetime} we mean the time an artifact will be in the same state as the state used by the authors.

As we have seen in Section\ \ref{sec:sop}, not all the popular tools and methods to share source code, package software environment, or platforms to perform experiments on have the same \emph{lifetime} guarantees/quality.

Artifacts with long \emph{lifetime} are more valuable and impactful for future researchers to extend, and deserve to be rewarded and have more visibility.
Table\ \ref{tab:lifetime} proposes a grading framework to evaluate the \emph{lifetime} of an artifact.
The artifact is evaluated on three criteria: sharing of the source code, experimental setup used, and software environment. 
We propose of each criteria 4 levels going from poor but passable quality (1/4) to \todo{perfect} (4/4).
Performing the average of the the grade for each criteria gives a global grade for the artifact.
We recommend for the global grade to be \emph{strictly} greater than 3/4 to deliver the \emph{lifetime} badge.


\begin{figure}
  \centering
    \includegraphics[width=0.5\textwidth]{figs/lifetime_score}
  \caption{Lifetime score for each of the three criteria, and the global lifetime score. We recommend a minimum lifetime score of 3 to award the lifetime badge. If we compute the lifetime score on the survey papers of Section \ref{sec:sop}, then about 1 percent would have get the badge.}\label{fig:lifetime_score}
\end{figure}

Figure \ref{fig:lifetime_score} shows the score for each of the three criteria (top 3 plots), and the global lifetime score (bottom plot) for the papers surveyed in Section\ \ref{sec:sop}.
With the recommended threshold of 3 out of 4, then about 1 percent of the survey papers would have get the badge.
We can see that the most penalizing criteria is the one about software environment, where the majority of the papers have zero out of four.

\todo{more discussion on the purpose of badges: rewarding effort, but it could also take a descision in the accept/reject process}


\section{Conclusion}

\todo{summarize contribs}

The artifact review should not only make sure that the work presented in the paper is reproducible, but should also make sure that the work could be reused by others in the future.
This means that the sources, data, should be available, etc.

the report of the \aeval\ should be linked to the paper.
will be done in SC24

What is the future of \aeval?
should the \aeval\ be part of the accept/reject decision process?
What about the energy/environmental cost of \aeval?
especially in HPC where the experiments are long lasting and resources consumming.
If the current \aeval\ is the first step towards a more "important" reviewing process, the community should not get used to this level of rigourness for \aeval.

One perspective is to make even more apparent the lack of reproducibility of the popular methods to generate and package software environment.
By collecting \texttt{Dockerfile}s from the artifacts of the papers, we could try to rebuild the Docker images from source periodically (\eg\ every month), and log the versions of the softwares in the resulting image.
As \texttt{Dockerfile} recipes mostly rely on either nvidia or Ubuntu based images, calls to \texttt{apt}, and \texttt{pip}, the resulting software environment is very fragile, and would thus be intersting to follow its evolution. 
This does not mean trying to rerun the experiments associated with the Docker images as it would be to energy consumming.
So, even if the software environment varies, it does not mean that the results of the experiments will vary.

\section*{Acknowledgments}

\todo{Thanks DAPHNE}

%% PAPER ENDS HERE -----------------------------------------------------------------------------------------------

%\bibliographystyle{sty/ACM-Reference-Format}
%\bibliography{references}
\printbibliography

\end{document}
