% This is samplepaper.tex, a sample chapter demonstrating the
% LLNCS macro package for Springer Computer Science proceedings;
% Version 2.21 of 2022/01/12
%
% \documentclass[runningheads]{llncs}
\documentclass[a4paper,table]{article}
\usepackage{authblk}

\usepackage[margin=20mm]{geometry}
%
\usepackage[T1]{fontenc}
% T1 fonts will be used to generate the final print and online PDFs,
% so please use T1 fonts in your manuscript whenever possible.
% Other font encondings may result in incorrect characters.
%
\usepackage[backend=bibtex, style=numeric]{biblatex}
% \usepackage[backend=biber, style=lncs]{biblatex}
\addbibresource{$bibliography$}

$for(header-includes)$
$header-includes$
$endfor$

\usepackage[table]{xcolor}
\usepackage{booktabs}
\usepackage{amsmath,amssymb,amsfonts}
\usepackage{algorithmic}
% \usepackage{graphicx}
\usepackage{textcomp}
\usepackage{hyperref}
\usepackage{longtable}
\def\BibTeX{{\rm B\kern-.05em{\sc i\kern-.025em b}\kern-.08em
    T\kern-.1667em\lower.7ex\hbox{E}\kern-.125emX}}

\usepackage{tcolorbox}
\tcbuselibrary{theorems}
\usepackage{cleveref}

\newtcbtheorem[]{trap}{Pitfall}{colback=black!5,colframe=black!35,fonttitle=\bfseries}{th}

\newcommand{\repro}{reproducibility}
\newcommand{\Repro}{Reproducibility}
\newcommand{\transpo}{\emph{Transposition}}
\newcommand{\flavour}{\emph{flavour}}
\newcommand{\flavours}{\emph{flavours}}
\newcommand{\ie}{\emph{i.e.,}}
\newcommand{\eg}{\emph{e.g.,}}
\newcommand{\nix}{\emph{Nix}}
\newcommand{\nixos}{\emph{NixOS}}
\newcommand{\nxc}{\emph{NixOS Compose}}
\newcommand{\enos}{\emph{EnOSlib}}
\newcommand{\grid}{\emph{Grid'5000}}
\newcommand{\kam}{\emph{Kameleon}}
\newcommand{\kad}{\emph{Kadeploy}}
\newcommand{\mel}{\emph{Melissa}}
\newcommand{\store}{\emph{Nix Store}}

% Used for displaying a sample figure. If possible, figure files should
% be included in EPS format.
%
% If you use the hyperref package, please uncomment the following two lines
% to display URLs in blue roman font according to Springer's eBook style:
%\usepackage{color}
%\renewcommand\UrlFont{\color{blue}\rmfamily}
%
\begin{document}
%
\title{$title$}
%
%\titlerunning{Abbreviated paper title}
% If the paper title is too long for the running head, you can set
% an abbreviated paper title here
%
% \author{
% $for(author)$
% 	% $author.name$\inst{$author.affiliation$}\orcidID{$author.orcid$}
% 	$author.name$%\orcidID{$author.orcid$}
%     % $sep$ \and
% $endfor$
% }
$for(author)$
	% $author.name$\inst{$author.affiliation$}\orcidID{$author.orcid$}
\author{
	$author.name$%\orcidID{$author.orcid$}
}

    % $sep$ \and
$endfor$

\affil{
$for(affiliation)$
$affiliation.institute$\footnote{\texttt{$affiliation.email$}}
$endfor$
}

% \institute{
% $for(affiliation)$
% $affiliation.institute$\\
% \email{$affiliation.email$}
% $sep$ \and
% $endfor$
% }

\date{}

% First names are abbreviated in the running head.
% If there are more than two authors, 'et al.' is used.
%
\maketitle              % typeset the header of the contribution
%
\begin{abstract}
$abstract$
\end{abstract}

% \keywords{%
% $for(keywords)$
%     $keywords$ $sep$ \and
% $endfor$
% }

$body$

\printbibliography
%
%
%
\end{document}
